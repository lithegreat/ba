\section{Evaluation}

\subsection{Setup}

For the evaluation of custom operations within the OpenASIP 2.0 Co-Design toolchain, we used the ETISS (Extendable Translating Instruction Set Simulator) integrated with the CoreDSL ecosystem. The primary benchmark used in this study was \texttt{coremark}, a widely-used benchmark for evaluating embedded systems' performance. ETISS was configured with the custom instructions generated from the OpenASIP operations, and multiple runs were performed to gather data on execution behavior.

Key parameters for the evaluation included the total number of instructions executed, ROM and RAM usage, and total clock cycles relative to the baseline configuration. Each test run was configured with specific options in the \texttt{coremark} benchmark, and additional logging features were enabled to capture detailed instruction-level analysis.

\subsection{Performance}

The table below shows key performance metrics gathered during the evaluation:

\begin{table}[h!]
    \centering
    \begin{tabular}{|c|c|c|c|c|c|}
        \hline
        \textbf{Session} & \textbf{Run} & \textbf{Total Instructions} & \textbf{Total ROM} & \textbf{Total RAM} & \textbf{Total Cycles (rel.)} \\
        \hline
        50  & 0 & 3,877,474 & 50,136 & 4,796 & 1.000 \\
        50  & 1 & 3,805,746 & 50,120 & 4,796 & 0.982 \\
        \hline
    \end{tabular}
    \caption{Summary of Performance Metrics for Coremark on ETISS}
\end{table}

Additionally, the custom OpenASIP operations executed during the benchmark include \texttt{openasip\_base\_shl1add}, \texttt{openasip\_base\_mac}, \texttt{openasip\_base\_shl2add}, \texttt{openasip\_base\_lt}, and \texttt{openasip\_base\_maxu}. The number of times each operation was executed and its relative impact on overall performance is shown below:

\begin{table}[h!]
    \centering
    \begin{tabular}{|l|c|c|}
        \hline
        \textbf{Operation} & \textbf{Executions} & \textbf{Relative Usage (\%)} \\
        \hline
        openasip\_base\_shl1add & 60,503 & 1.6\% \\
        openasip\_base\_mac     & 58,347 & 1.5\% \\
        openasip\_base\_shl2add & 19,876 & 0.5\% \\
        openasip\_base\_lt      & 12,960 & 0.3\% \\
        openasip\_base\_maxu    & 41     & 0.0\% \\
        \hline
    \end{tabular}
    \caption{Execution Count of Custom OpenASIP Operations}
\end{table}

The results show that the custom operations contributed a small but measurable portion to the overall execution, with \texttt{openasip\_base\_shl1add} and \texttt{openasip\_base\_mac} being the most frequently used operations.

\subsection{Discussion}

The evaluation reveals that the custom OpenASIP instructions were successfully integrated into the ETISS framework and executed as part of the \texttt{coremark} benchmark. The performance metrics indicate a slight reduction in total instructions and clock cycles between the two test runs, with a relative cycle decrease of approximately 1.8\%.

The custom instructions, such as \texttt{openasip\_base\_shl1add} and \texttt{openasip\_base\_mac}, contributed to specific operations in the benchmark, especially in areas involving shifts and multiplications. These operations, while not heavily used, offered opportunities for optimization in terms of ROM and RAM usage, particularly for workloads that can benefit from these types of operations.

While the impact of the custom instructions on overall performance was modest in this case, the results demonstrate the potential for further optimization. Future work could involve refining the operation set and targeting specific use cases where these custom instructions can lead to more significant performance gains. Additionally, more complex machine learning models could be evaluated to better understand the scalability and broader applicability of these operations.

In summary, this evaluation highlights the viability of integrating custom instructions from the OpenASIP framework into the ETISS simulation environment and provides a foundation for further exploration into custom instruction optimizations for embedded systems.
