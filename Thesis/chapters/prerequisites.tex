To successfully port OpenASIP custom operations to CoreDSL and integrate them into ETISS,
it is essential to understand how OpenASIP defines operations using trigger code and XML representation.
This section outlines the key aspects of these definitions and their conversion requirements.

OpenASIP provides a framework for defining custom instructions using two main components: trigger code and XML representation.
Understanding these components is crucial for translating OpenASIP instructions into CoreDSL format.

\section{Trigger Code}

In OpenASIP, the behavior of custom operations is defined using trigger code. This code specifies how inputs are processed to produce outputs.
The trigger code is written in a domain-specific language tailored for describing operational logic.
Below is an example of trigger code for an integer addition operation:

\begin{lstlisting}[caption={The trigger code of ADD}]
  // ADD - integer add
  OPERATION(ADD)

  TRIGGER
      IO(3) = UINT(1) + UINT(2);
  END_TRIGGER;

  END_OPERATION(ADD)
\end{lstlisting}

\begin{itemize}
  \item \texttt{OPERATION(ADD)} begins the definition of the ADD operation.
  \item The \texttt{TRIGGER} block contains the logic where \texttt{IO(3)} is assigned the result of adding \texttt{IO(1)} and \texttt{IO(2)}.
  \item \texttt{END\_OPERATION(ADD)} concludes the operation definition.
\end{itemize}

The trigger code provides a concise way to define the logic for processing inputs and generating outputs.
This code needs to be translated into CoreDSL syntax for simulation in ETISS.

\section{XML Representation}

OpenASIP also uses XML to define the properties and semantics of custom operations.
The XML representation describes the operation's inputs, outputs, and execution semantics.
Here is an example of how the ADD operation is represented in XML:

\begin{lstlisting}[caption={The XML representation of ADD}]
  <operation>
    <name>ADD</name>
    <description>Integer addition. Output 3 is sum of inputs 1 and 2.</description>
    <inputs>2</inputs>
    <outputs>1</outputs>
    <in element-count="1" element-width="32" id="1" type="SIntWord">
      <can-swap>
        <in id="2"/>
      </can-swap>
    </in>
    <in element-count="1" element-width="32" id="2" type="SIntWord">
      <can-swap>
        <in id="1"/>
      </can-swap>
    </in>
    <out element-count="1" element-width="32" id="3" type="SIntWord"/>
    <trigger-semantics>
      EXEC_OPERATION(add, IO(1), IO(2), IO(3));
    </trigger-semantics>
  </operation>
\end{lstlisting}

In this XML representation:
\begin{itemize}
  \item \texttt{<name>} specifies the operation's name (ADD).
  \item \texttt{<description>} provides a textual explanation of the operation.
  \item \texttt{<inputs>} and \texttt{<outputs>} define the number of inputs and outputs.
  \item \texttt{<in>} and \texttt{<out>} elements describe the details of each input and output, such as width and type.
  \item \texttt{<trigger-semantics>} provides the execution logic for the operation, mapping directly to the trigger code.
\end{itemize}

The XML representation captures the essential properties and semantics of custom operations,
which are necessary for filtering the operations.
