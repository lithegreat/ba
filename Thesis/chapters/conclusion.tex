\section{Conclusion}

In this work, we evaluated the integration and performance of custom instructions from the OpenASIP 2.0 Co-Design toolchain within the ETISS simulation framework. Using the CoreDSL ecosystem, we successfully translated custom OpenASIP operations into a format compatible with ETISS and tested these instructions using the \texttt{coremark} benchmark.

The evaluation demonstrated that the custom instructions, while not dominating the overall execution, contributed specific enhancements in operations such as shifts and multiplications. Although the relative performance gains in terms of reduced instruction count and clock cycles were modest, this study highlights the potential for custom instructions to optimize embedded system performance, particularly in applications that can exploit specific custom operations.

Furthermore, the use of M2-ISA-R as a testing and debugging tool proved invaluable in identifying and correcting errors during the code generation process, allowing for an iterative development approach that ensured compatibility with the ETISS platform. The seamless integration of custom instructions and the observed performance metrics suggest that, with further refinement and targeting of specific workloads, significant optimizations can be achieved.

Future work will focus on extending the range of custom operations and evaluating their impact on more complex machine learning models and workloads. Additionally, more aggressive optimizations in both the instruction set and the ETISS simulation platform can be explored to maximize the performance benefits of custom instruction sets for embedded applications.

In summary, this study provides a foundational step toward enhancing embedded system performance through custom ISA extensions, and demonstrates the practicality of using tools like M2-ISA-R and ETISS to evaluate and benchmark these enhancements.