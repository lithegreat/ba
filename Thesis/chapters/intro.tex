RISC-V, an open-source instruction set architecture (ISA), has gained significant traction across various domains due to its modular design and open nature.
Its applications are diverse, ranging from embedded systems powering sensors and microcontrollers to consumer electronics like smartphones and smartwatches.
In automotive technology, RISC-V is increasingly integrated into advanced driver-assistance systems (ADAS) and infotainment units.
Its flexibility also makes it a popular choice in academic research for processor design and computer architecture experimentation.
This versatility and open-source model drive innovation and customization across numerous technological sectors. \cite{Risc-v}

To fully leverage RISC-V's flexibility, effective simulation and validation of custom instructions are essential.
The ETISS (Extendable Translating Instruction Set Simulator) \cite{ETISS} offers a robust platform for simulating and testing custom processor architectures.
However, integrating ETISS with a tool that allows for flexible definition and specification of custom instructions can enhance its capabilities.

OpenASIP 2.0 \cite{OpenASIP} is a co-design toolset that facilitates the customization of RISC-V-based processors,
supporting RTL generation and high-level programming of custom instructions.
To improve the integration with ETISS, we translate OpenASIP's custom instruction sets into CoreDSL \cite{CoreDSL}—a descriptive language for hardware design.
CoreDSL enables clear and flexible specification of custom instructions, which can then be seamlessly applied to ETISS for comprehensive simulation and evaluation.

This paper focuses on this translation process, aiming to streamline the integration and evaluation of custom instructions.
Our approach enhances the effectiveness and efficiency of the RISC-V simulation environment,
driving more effective prototyping and validation of custom processor architectures.
