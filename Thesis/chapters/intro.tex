RISC-V, an open-source instruction set architecture (ISA), has seen widespread adoption across various domains due to its modular design and open nature.
Its applications span embedded systems, where it powers devices like sensors and microcontrollers, and consumer electronics, including smartphones and smartwatches.
In automotive technology, RISC-V is being integrated into advanced driver-assistance systems (ADAS) and infotainment units. Additionally,
its flexibility makes it a popular choice in academic research for processor design and computer architecture experimentation.
The versatility and open-source model of RISC-V facilitate innovation and customization, driving advancements in numerous technological sectors.

To leverage this flexibility effectively, simulation and validation of custom instructions are essential.
The ETISS (Extendable Translating Instruction Set Simulator) \cite{ETISS} provides a robust platform for instruction set simulation,
offering the capability to model and test custom processor architectures.
However, to fully utilize ETISS's capabilities, it is beneficial to integrate it with a flexible tool for defining custom instructions.

OpenASIP 2.0 \cite{OpenASIP} is a co-design toolset designed to facilitate the customization of RISC-V-based processors,
supporting RTL generation and high-level programming of custom instructions.
By translating OpenASIP's custom instruction sets into CoreDSL \cite{CoreDSL} syntax—a descriptive language for hardware design—we can enhance the integration with ETISS.
CoreDSL allows for a clear and flexible specification of custom instructions,
which can be seamlessly applied to ETISS for comprehensive simulation and evaluation.

Thus, this paper focuses on translating OpenASIP's custom instruction sets into CoreDSL language and applying them within the ETISS framework.
This approach aims to streamline the process of integrating and evaluating custom instructions,
ultimately improving the effectiveness and efficiency of the simulation environment for RISC-V processors.