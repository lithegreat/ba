The integration of custom instruction sets within application-specific instruction set processors (ASIPs) has become increasingly vital for enhancing performance in specialized applications, particularly in artificial intelligence (AI) and embedded systems. Recent studies underscore the importance of these custom instruction sets in addressing performance bottlenecks associated with memory access, which is critical for efficient operation in systems such as Deep Neural Networks (DNNs).

The work by Oh and Lee (2023) is the integration of custom instruction set extensions (ISE) within RISC architectures, enabling a more efficient uniprocessor system. This method minimizes communication overhead by embedding the AI processor into the general-purpose processor (GPP), allowing for on-chip data transfer that eliminates complex communication protocols. The proposed architecture, based on a 32-bit MIPS core, achieves remarkable improvements, reporting a throughput performance enhancement of up to 193.88 times and an energy efficiency increase of 52.75 times compared to traditional heterogeneous systems \cite{oh2023design}. This integrated design not only addresses the resource limitations of lightweight systems but also provides a scalable framework for future AI applications.


Kumar et al. (2024) further contribute to this discussion by exploring an ISA extension for RISC-V focused on optimizing memory load and store operations—key operations in many embedded applications, especially those involving DNNs. Their study demonstrates a novel instruction that allows for double-word memory access, resulting in substantial reductions in clock cycles (approximately 50\%) and power consumption (around 30\%) during these operations, while maintaining only a minimal area overhead of about 4\% on a modified RISC-V platform. This work validates the need for efficient memory interaction to mitigate power demands and latency, particularly in edge-AI applications where resource constraints are prevalent \cite{kumar2024implementation}.

Salim et al. (2012) present a comprehensive simulation framework that targets the reconfigurable architecture of an 8-bit PIC16C5X-compatible processor. By utilizing a Java-based simulation platform, the authors modify instruction widths and machine instructions while incorporating memory address remapping techniques to effectively address challenges related to memory banking schemes. This simulation process successfully verifies a total of 34 customized instruction sets, demonstrating the capability to adapt existing instructions to meet the specific needs of various applications. Notably, the paper highlights the modification of specific instructions such as "movf f,1" and "swapf f,1" to enhance their functionality, thereby showcasing the flexibility and robustness of the proposed simulation framework. The results indicate that this approach not only maintains compatibility with the original PIC instruction set architecture but also facilitates the implementation of tailored operations that can significantly improve performance in embedded systems. Thus, this work reinforces the viability of soft-core processors as platforms for developing custom architectures, aligning with the broader ASIP methodology \cite{salim2012customized}.

Chen et al. (2023) present a compelling solution by proposing lightweight custom instructions that target essential mathematical operations such as exponentials, logarithms, and trigonometric functions. These operations are critical for numerous IoT applications, including communications, image processing, and biomedical signal processing. Their research demonstrates that the implementation of these custom instructions can achieve impressive speedups ranging from 3.3 to 18.0 times compared to baseline designs based on the RV32IM architecture. Furthermore, the study highlights that the power overhead remains relatively low—under 5\% for the tiny variant, under 17\% for the intermediate variant, and under 26\% for the fast variant—thereby validating the efficiency of custom instruction sets within resource-constrained environments. Overall, this work not only addresses the computational inefficiencies associated with elementary function calculations in IoT devices but also emphasizes the versatility of the proposed instructions in adapting to the diverse performance, power, and area (PPA) requirements of various IoT applications \cite{chen2023risc}.

In light of these advancements, our study utilizes the OpenASIP 2.0 Co-Design toolchain to evaluate custom operations within the CoreDSL ecosystem. By translating OpenASIP custom operations into CoreDSL syntax and leveraging the Extendable Translating Instruction Set Simulator (ETISS), we benchmark the performance of these operations against MLonMCU benchmarks. The results indicate a promising moderate speedup, underscoring the potential of custom instruction sets to improve performance in modern embedded applications, akin to the benefits highlighted in the aforementioned research. This approach not only extends the current understanding of custom instruction integration but also provides a foundation for further exploration in the realm of performance optimization for specialized processors.