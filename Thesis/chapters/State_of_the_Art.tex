\section*{Background}

The RISC-V instruction set architecture (ISA) has gained prominence due to its open-source nature and flexibility, allowing for extensive customization.
This adaptability is particularly valuable in the development of application-specific instruction-set processors (ASIPs),
where processors are tailored to meet the specific needs of particular applications.

\section*{Instruction Set Simulation and Customization}

\textbf{ETISS} (Extendable Translating Instruction Set Simulator) \cite{ETISS} provides a comprehensive platform for simulating and testing custom processor architectures.
ETISS supports the modeling of both functional and performance aspects of processors, making it a robust tool for evaluating custom ISAs.
The simulator's extensibility allows for detailed exploration of custom instructions and their integration into different processor designs.

\textbf{OpenASIP 2.0} \cite{OpenASIP} is a co-design toolset designed to streamline the customization of RISC-V-based processors.
It supports the generation of Register Transfer Level (RTL) code and high-level programming of custom instructions.
OpenASIP 2.0 allows for the efficient integration of custom operations by providing a framework for defining and validating new instructions.

\textbf{CoreDSL} \cite{CoreDSL} is a descriptive language used for hardware design that facilitates the specification of custom instructions.
It allows designers to define instruction sets in a clear and flexible manner, which can then be used to generate RTL code or integrate with simulation tools like ETISS.

\section*{Current Advances}

Recent advancements in the field have focused on enhancing the flexibility and efficiency of instruction set simulation and customization.
Techniques for translating custom instructions into simulation environments have been developed to improve the accuracy and performance of evaluations.
For instance, translating custom instruction sets from tools like OpenASIP into CoreDSL syntax enables seamless integration with simulators such as ETISS,
allowing for comprehensive testing and validation.
