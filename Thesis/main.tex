%%
% This is an Overleaf template for Master's and Bachelor's theses
% using the TUM Corporate Design https://www.tum.de/cd
%
% For further details on how to use the template, take a look at our
% GitLab repository and browse through our test documents
% https://gitlab.lrz.de/latex4ei/tum-templates.
%
% The tumbook class is based on the KOMA-Script class scrbook.
% If you need further customization please consult the KOMA-Script guide
% https://ctan.org/pkg/koma-script.
% Additional class options are passed down to the base class.
%
% If you encounter any bugs or undesired behaviour, please raise an issue
% in our GitLab repository
% https://gitlab.lrz.de/latex4ei/tum-templates/issues
% and provide a description and minimal working example of your problem.
%%


\documentclass[
  a4paper,            % paper size (a4paper, a5paper)
  thesis=student,     % define the type of the thesis (student, phd, none)
  english,            % define the document language (english, german)
  % BCOR=5mm,           % define a binding offset for the document
  % coverBCOR=1cm,      % define a different binding offset for the cover page
  coverpage=false,    % disable the cover page (e.g. if a tumcover is used)
  titlepage=false,    % disable the additional title page
  twoside,            % use onesided or twosided layout (oneside, twoside)
  % headmarks=true,     % enable headmarks (true, false)
  font=times          % define main text font (helvet, times, palatino, libertine)
]{tumbook}

% For theses that are printed with a transparent cover it is recommended to
% use the coverpage and provide the proper coverBCOR, so the distance between
% the binding strip and the content is properly set to 1 * logoheight.
% In this case, the publisher and titleback information is most certainly
% empty and the titlepage may be turned off.
%
% For theses that are printed with a soft cover or by a publisher it is
% recommended to create a cover using the tumcover class and therefore turn
% off the coverpage here. In this case, you most certainly have publisher and
% titleback information and you should keep the titlepage option enabled.


% load additional packages
\usepackage{lipsum}


% thesis metadata
\title{Evaluation of OpenASIP Custom Operations in CoreDSL Ecosystem}
\subtitle{}
\author{Hengsheng Li}

\degree{Bachelor of Science (B.Sc.)}
\dateSubmitted{01.10.2024}

\examiner{Prof.\@~Dr.-Ing.\@ Ulf Schlichtmann}
\supervisor{M.Sc. Philipp van Kempen}


\begin{document}

\frontmatter
\maketitle
\chapter{Abstract}
This thesis evaluates the integration and performance improvement of custom operations designed using the OpenASIP 2.0 Co-Design toolchain within the CoreDSL ecosystem, leveraging the Extendable Translating Instruction Set Simulator (ETISS). As hardware-software co-design becomes increasingly crucial in the optimization of modern embedded systems, custom instructions tailored to specific applications can significantly enhance performance. This research systematically collected and translated OpenASIP-defined custom operations into CoreDSL syntax to create a pipeline for generating ETISS-compatible architectures. These architectures were evaluated against a series of benchmarks using the MLonMCU framework, a tool designed for benchmarking machine learning workloads on microcontrollers. In this study, we delve into the intricacies of translating complex custom operations and the corresponding challenges within the CoreDSL environment, examining the effectiveness of ETISS in simulating these architectures. Through comprehensive benchmarking, we demonstrate that custom instructions, when effectively integrated, can yield moderate performance improvements over the baseline, reflecting the potential of custom operations in enhancing application-specific workloads.
\tableofcontents

\mainmatter
\chapter{Introduction}
RISC-V, an open-source instruction set architecture (ISA), has seen widespread adoption across various domains due to its modular design and open nature.
Its applications span embedded systems, where it powers devices like sensors and microcontrollers, and consumer electronics, including smartphones and smartwatches.
In automotive technology, RISC-V is being integrated into advanced driver-assistance systems (ADAS) and infotainment units. Additionally,
its flexibility makes it a popular choice in academic research for processor design and computer architecture experimentation.
The versatility and open-source model of RISC-V facilitate innovation and customization, driving advancements in numerous technological sectors.

To leverage this flexibility effectively, simulation and validation of custom instructions are essential.
The ETISS (Extendable Translating Instruction Set Simulator) \cite{ETISS} provides a robust platform for instruction set simulation,
offering the capability to model and test custom processor architectures.
However, to fully utilize ETISS's capabilities, it is beneficial to integrate it with a flexible tool for defining custom instructions.

OpenASIP 2.0 \cite{OpenASIP} is a co-design toolset designed to facilitate the customization of RISC-V-based processors,
supporting RTL generation and high-level programming of custom instructions.
By translating OpenASIP's custom instruction sets into CoreDSL \cite{CoreDSL} syntax—a descriptive language for hardware design—we can enhance the integration with ETISS.
CoreDSL allows for a clear and flexible specification of custom instructions,
which can be seamlessly applied to ETISS for comprehensive simulation and evaluation.

Thus, this paper focuses on translating OpenASIP's custom instruction sets into CoreDSL language and applying them within the ETISS framework.
This approach aims to streamline the process of integrating and evaluating custom instructions,
ultimately improving the effectiveness and efficiency of the simulation environment for RISC-V processors.

\chapter{State of the Art}
The integration of custom instruction sets within application-specific instruction set processors (ASIPs) has become increasingly vital for enhancing performance in specialized applications, particularly in artificial intelligence (AI) and embedded systems. Recent studies underscore the importance of these custom instruction sets in addressing performance bottlenecks associated with memory access, which is critical for efficient operation in systems such as Deep Neural Networks (DNNs).

The work by Oh and Lee (2023) highlights the role of customizable instruction set architectures (ISAs) in reducing communication overhead and improving energy efficiency by integrating AI processors within general-purpose processors (GPPs) \cite{oh2023design}. This seamless integration allows for significant enhancements in throughput while minimizing the complexity typically associated with heterogeneous computing systems.

Kumar et al. (2024) further contribute to this discussion by exploring an ISA extension for RISC-V focused on optimizing memory load and store operations—key operations in many embedded applications, especially those involving DNNs. Their study demonstrates a novel instruction that allows for double-word memory access, resulting in substantial reductions in clock cycles (approximately 50\%) and power consumption (around 30\%) during these operations, while maintaining only a minimal area overhead of about 4\% on a modified RISC-V platform. This work validates the need for efficient memory interaction to mitigate power demands and latency, particularly in edge-AI applications where resource constraints are prevalent \cite{kumar2024implementation}.

Salim et al. (2012) also provide valuable insights into the effectiveness of custom instruction sets in RISC processors. Their research illustrates how modifying instruction set architectures can enhance the flexibility and capabilities of embedded systems, reinforcing the necessity of addressing memory organization and physical address remapping to optimize performance \cite{salim2012customized}.

In light of these advancements, our study utilizes the OpenASIP 2.0 Co-Design toolchain to evaluate custom operations within the CoreDSL ecosystem. By translating OpenASIP custom operations into CoreDSL syntax and leveraging the Extendable Translating Instruction Set Simulator (ETISS), we benchmark the performance of these operations against MLonMCU benchmarks. The results indicate a promising moderate speedup, underscoring the potential of custom instruction sets to improve performance in modern embedded applications, akin to the benefits highlighted in the aforementioned research. This approach not only extends the current understanding of custom instruction integration but also provides a foundation for further exploration in the realm of performance optimization for specialized processors.

\chapter{Prerequisites}
\section{LLVM}

LLVM (Low-Level Virtual Machine) is a versatile compiler infrastructure that supports the compilation of various programming languages to multiple target architectures. Its architecture consists of a collection of reusable compiler and toolchain technologies, providing a robust framework for both frontend and backend development.

At its core, LLVM employs an intermediate representation (IR) that serves as a common code representation between source languages and target architectures. This IR is designed to be language-agnostic and can be optimized independently of the source code, enabling a wide range of optimization techniques to be applied uniformly across different languages.

One of the key features of LLVM is its modularity, which allows developers to extend the compiler with new optimizations or target architectures without modifying existing components. This flexibility is particularly advantageous for research and development in custom instruction sets, facilitating rapid prototyping and experimentation.

The compilation process in LLVM typically comprises several stages:

\begin{enumerate}
    \item \textbf{Frontend:} This component parses the source code and converts it into LLVM IR. Frontends can be developed for various programming languages, allowing LLVM to serve as a backend for diverse ecosystems.

    \item \textbf{Optimization:} Once in IR form, LLVM applies a series of optimization passes that enhance performance, reduce code size, and improve overall efficiency. These optimizations can be tailored based on the specific characteristics of the target architecture.

    \item \textbf{Backend:} The backend is responsible for generating machine code specific to the target architecture. It leverages target-specific information to produce optimized and efficient binaries.
\end{enumerate}

In the context of custom instruction sets, LLVM's design allows for seamless integration of vendor-defined extensions. By defining new instruction patterns and leveraging existing optimization frameworks, developers can enhance the performance of their applications without requiring deep knowledge of the underlying compiler infrastructure \cite{llvm}.

Overall, LLVM provides a powerful platform for compiling and optimizing code, making it an ideal choice for implementing and exploring custom ISAs. The capabilities of LLVM, particularly its support for autovectorization and advanced optimization techniques, directly align with the objectives of this research, facilitating the effective utilization of custom instruction sets in performance-critical applications.


\section{M2-ISA-R}
M2-ISA-R is a framework designed to facilitate the description and implementation of custom ISAs. It uses a metamodel-based approach that allows for a high level of abstraction when specifying both functional and structural components of a processor architecture. This flexibility in modeling makes it particularly useful for hardware-software co-design.

The workflow of M2-ISA-R involves parsing the ISA description via a frontend that utilizes the ANTLR4 framework. It then generates Python-based metamodel classes to capture both the architectural and behavioral aspects of the processor. The backend produces architecture-specific plugins for simulation tools, the most important of which is the ETISS architecture plugin. This enables rapid creation of CPU models for ETISS, closing the gap between high-level ISA descriptions and functional simulation.

In this research, M2-ISA-R has been employed to automatically generate the ETISS CPU architecture plugin for a custom ISA, allowing the seamless integration of this architecture into the ETISS simulation framework. This significantly reduces the development time needed for functional verification and performance simulation of custom CPU designs \cite{RISCVSimulation}.

\section{Seal5}

This work builds upon the Seal5 framework, which provides a semi-automated flow for generating LLVM compiler support for custom RISC-V instruction set architectures (ISAs). Seal5 leverages a C-style ISA description language to facilitate the integration of vendor-defined instructions, allowing for efficient exploration and implementation of custom instructions tailored to specific applications.

Seal5 enables the automatic generation of LLVM patches that cover a wide range of functionalities, from baseline assembly-level support to sophisticated compiler code generation patterns for scalar and vector instructions. The tool features a novel pattern generator approach focused on optimizing code generation for SIMD (Single Instruction, Multiple Data) instructions, including autovectorization capabilities.

By utilizing Seal5, this research aims to convert a custom instruction set into a customized LLVM backend. This process significantly reduces development time and effort while achieving performance levels comparable to or exceeding those of manually implemented LLVM toolchains. The ability of Seal5 to support autovectorization enhances the execution efficiency of workloads, making it an invaluable asset for the exploration of custom ISA extensions \cite{Seal5}.

\chapter*{Implementation}
\section{Implementation Workflow}

The workflow for generating CoreDSL code from OSAL files, as depicted in Figure 4.1, involves several key stages that transform input operations into CoreDSL code suitable for integration into M2-ISA-R. The overall flow can be divided into the following steps:

\begin{enumerate}
    \item \textbf{Parsing the OSAL Files:} The process begins with the input OSAL (Operation Set Architecture Language) files, which contain the operation definitions. These files are processed by an XML Parser that extracts the operation properties and relevant data.

    \item \textbf{Filtering Operations:} Once parsed, the operations undergo a filtering stage. Here, multiple criteria are applied, such as the number of inputs and outputs, whether the operation has side effects, and other predefined properties. These conditions help identify which operations are suitable for further processing.

    \item \textbf{Handling Single Execution Operations:} If the operation is determined to be a single execution type (i.e., simple operations that can be executed in a single step), it is immediately passed to the transformation stage. For non-single execution operations, a base operation is identified and combined with other operations as needed before continuing the process.

    \item \textbf{Transforming to CoreDSL:} The filtered and combined operations are transformed into CoreDSL. This step includes encoding the behavioral aspects of the operation into a format compatible with the CoreDSL language, facilitating its integration into the larger ISA model.

    \item \textbf{Auto Encoding and Assembly:} Alongside the behavioral encoding, the process includes an automated stage for encoding and assembly generation. This ensures that the operations are fully represented in both the abstract behavioral sense and in terms of the machine-level encoding required for execution.

    \item \textbf{Code Generation:} A CoreDSL template is used to guide the generation of the final code. The CoreDSL template serves as a scaffold, ensuring consistency and completeness in the generated output.

    \item \textbf{Final Code Writing:} The processed operations, including the behavior and encoding information, are passed to the writer. The writer generates the final CoreDSL code, which represents the operations in a format that is ready for integration into M2-ISA-R.

\end{enumerate}

This workflow ensures a structured and efficient transformation of OSAL files into CoreDSL code, leveraging automated encoding and assembly processes to streamline the generation of instruction set architectures for custom CPU designs.

\begin{figure}[h]
  \centering
  \includegraphics[width=\linewidth]{figures/flow.png}
  \caption{Implementation Flow}
\end{figure}

\section{XML Parser}

As mentioned in the previous section, OpenASIP uses XML to define the properties and semantics of custom operations.
To translate these operations into CoreDSL, we need to parse the XML representation and extract the necessary information.
We developed an XML parser based on pandas package in Python to read the XML files and extract the operation details.


\section{Operation Filtering}

\section{CoreDSL Template Generation}

\section{Behavior Code Generation}

\chapter{Results}
\section{Evaluation}

\subsection{Setup}

For the evaluation of custom operations within the OpenASIP 2.0 Co-Design toolchain, we used the ETISS (Extendable Translating Instruction Set Simulator) integrated with the CoreDSL ecosystem. The primary benchmark used in this study was \texttt{coremark}, a widely-used benchmark for evaluating embedded systems' performance. ETISS was configured with the custom instructions generated from the OpenASIP operations, and multiple runs were performed to gather data on execution behavior.

Key parameters for the evaluation included the total number of instructions executed, ROM and RAM usage, and total clock cycles relative to the baseline configuration. Each test run was configured with specific options in the \texttt{coremark} benchmark, and additional logging features were enabled to capture detailed instruction-level analysis.

\subsection{Performance}

The table below shows key performance metrics gathered during the evaluation:

\begin{table}[h!]
    \centering
    \begin{tabular}{|c|c|c|c|c|c|}
        \hline
        \textbf{Session} & \textbf{Run} & \textbf{Total Instructions} & \textbf{Total ROM} & \textbf{Total RAM} & \textbf{Total Cycles (rel.)} \\
        \hline
        50  & 0 & 3,877,474 & 50,136 & 4,796 & 1.000 \\
        50  & 1 & 3,805,746 & 50,120 & 4,796 & 0.982 \\
        \hline
    \end{tabular}
    \caption{Summary of Performance Metrics for Coremark on ETISS}
\end{table}

Additionally, the custom OpenASIP operations executed during the benchmark include \texttt{openasip\_base\_shl1add}, \texttt{openasip\_base\_mac}, \texttt{openasip\_base\_shl2add}, \texttt{openasip\_base\_lt}, and \texttt{openasip\_base\_maxu}. The number of times each operation was executed and its relative impact on overall performance is shown below:

\begin{table}[h!]
    \centering
    \begin{tabular}{|l|c|c|}
        \hline
        \textbf{Operation} & \textbf{Executions} & \textbf{Relative Usage (\%)} \\
        \hline
        openasip\_base\_shl1add & 60,503 & 1.6\% \\
        openasip\_base\_mac     & 58,347 & 1.5\% \\
        openasip\_base\_shl2add & 19,876 & 0.5\% \\
        openasip\_base\_lt      & 12,960 & 0.3\% \\
        openasip\_base\_maxu    & 41     & 0.0\% \\
        \hline
    \end{tabular}
    \caption{Execution Count of Custom OpenASIP Operations}
\end{table}

The results show that the custom operations contributed a small but measurable portion to the overall execution, with \texttt{openasip\_base\_shl1add} and \texttt{openasip\_base\_mac} being the most frequently used operations.

\subsection{Discussion}

The evaluation reveals that the custom OpenASIP instructions were successfully integrated into the ETISS framework and executed as part of the \texttt{coremark} benchmark. The performance metrics indicate a slight reduction in total instructions and clock cycles between the two test runs, with a relative cycle decrease of approximately 1.8\%.

The custom instructions, such as \texttt{openasip\_base\_shl1add} and \texttt{openasip\_base\_mac}, contributed to specific operations in the benchmark, especially in areas involving shifts and multiplications. These operations, while not heavily used, offered opportunities for optimization in terms of ROM and RAM usage, particularly for workloads that can benefit from these types of operations.

While the impact of the custom instructions on overall performance was modest in this case, the results demonstrate the potential for further optimization. Future work could involve refining the operation set and targeting specific use cases where these custom instructions can lead to more significant performance gains. Additionally, more complex machine learning models could be evaluated to better understand the scalability and broader applicability of these operations.

In summary, this evaluation highlights the viability of integrating custom instructions from the OpenASIP framework into the ETISS simulation environment and provides a foundation for further exploration into custom instruction optimizations for embedded systems.


\chapter{Discussion}
\input{chapters/discussion.tex}

\chapter{Conclusion}
\section{Conclusion}

In this work, we evaluated the integration and performance of custom instructions from the OpenASIP 2.0 Co-Design toolchain within the ETISS simulation framework. Using the CoreDSL ecosystem, we successfully translated custom OpenASIP operations into a format compatible with ETISS and tested these instructions using the MLonMCU benchmark suite.

The evaluation demonstrated that the custom instructions, while not dominating the overall execution, contributed specific enhancements in operations such as shifts and multiplications. Although the relative performance gains in terms of reduced instruction count and clock cycles were modest, this study highlights the potential for custom instructions to optimize embedded system performance, particularly in applications that can exploit specific custom operations.

Furthermore, the use of M2-ISA-R as a testing and debugging tool proved invaluable in identifying and correcting errors during the code generation process, allowing for an iterative development approach that ensured compatibility with the ETISS platform. The seamless integration of custom instructions and the observed performance metrics suggest that, with further refinement and targeting of specific workloads, significant optimizations can be achieved.

Future work will focus on extending the range of custom operations and evaluating their impact on more complex machine learning models and workloads. Additionally, more aggressive optimizations in both the instruction set and the ETISS simulation platform can be explored to maximize the performance benefits of custom instruction sets for embedded applications.

In summary, this study provides a foundational step toward enhancing embedded system performance through custom ISA extensions, and demonstrates the practicality of using tools like M2-ISA-R and ETISS to evaluate and benchmark these enhancements.

\appendix
\chapter{Appendix}
\lipsum[4]

\backmatter
\bibliographystyle{plain}
\bibliography{chapters/references}

\end{document}
