%%
% This is an Overleaf template for Master's and Bachelor's theses
% using the TUM Corporate Desing https://www.tum.de/cd
%
% For further details on how to use the template, take a look at our
% GitLab repository and browse through our test documents
% https://gitlab.lrz.de/latex4ei/tum-templates.
%
% The tumbook class is based on the KOMA-Script class scrbook.
% If you need further customization please consult the KOMA-Script guide
% https://ctan.org/pkg/koma-script.
% Additional class options are passed down to the base class.
%
% If you encounter any bugs or undesired behaviour, please raise an issue
% in our GitLab repository
% https://gitlab.lrz.de/latex4ei/tum-templates/issues
% and provide a description and minimal working example of your problem.
%%


\documentclass[
  a4paper,            % paper size (a4paper, a5paper)
  thesis=student,     % define the type of the thesis (student, phd, none)
  english,            % define the document language (english, german)
  % BCOR=5mm,           % define a binding offset for the document
  % coverBCOR=1cm,      % define a different binding offset for the cover page
  % coverpage=false,    % disable the cover page (e.g. if a tumcover is used)
  % titlepage=false,    % disable the additional title page
  % oneside,            % use onesided or twosided layout (oneside, twoside)
  % headmarks=true,     % enable headmarks (true, false)
  % font=times          % define main text font (helvet, times, palatino, libertine)
]{tumbook}

% For theses that are printed with a transparent cover it is recommended to
% use the coverpage and provide the proper coverBCOR, so the distance between
% the binding strip and the content is properly set to 1 * logoheight.
% In this case, the publisher and titleback information is most certainly
% empty and the titlepage may be turned off.
%
% For theses that are printed with a soft cover or by a publisher it is
% recommended to create a cover using the tumcover class and therefore turn
% off the coverpage here. In this case, you most certainly have publisher and
% titleback information and you should keep the titlepage option enabled.


% load additional packages
\usepackage{lipsum}


% thesis metadata
\title{Evaluation of OpenASIP Custom Operations in CoreDSL Ecosystem}
\subtitle{Porting OpenASIP Custom Operations to CoreDSL Syntax}
\author{Hengsheng Li}

\degree{Bachelor of Science (B.Sc.)}
\dateSubmitted{01.10.2024}

\examiner{Prof.\@~Dr.-Ing.\@ Ulf Schlichtmann}
\supervisor{van Kempen, Philipp; M.Sc.}


\begin{document}

\frontmatter
\maketitle
\chapter{Abstract}
This thesis evaluates the integration and performance improvement of custom operations designed using the OpenASIP 2.0 Co-Design toolchain within the CoreDSL ecosystem, leveraging the Extendable Translating Instruction Set Simulator (ETISS). As hardware-software co-design becomes increasingly crucial in the optimization of modern embedded systems, custom instructions tailored to specific applications can significantly enhance performance. This research systematically collected and translated OpenASIP-defined custom operations into CoreDSL syntax to create a pipeline for generating ETISS-compatible architectures. These architectures were evaluated against a series of benchmarks using the MLonMCU framework, a tool designed for benchmarking machine learning workloads on microcontrollers. In this study, we delve into the intricacies of translating complex custom operations and the corresponding challenges within the CoreDSL environment, examining the effectiveness of ETISS in simulating these architectures. Through comprehensive benchmarking, we demonstrate that custom instructions, when effectively integrated, can yield up to 28\% performance improvements over the baseline, reflecting the potential of custom operations in enhancing application-specific workloads.
\tableofcontents

\mainmatter
\chapter{Introduction}
RISC-V, an open-source instruction set architecture (ISA), has seen widespread adoption across various domains due to its modular design and open nature.
Its applications span embedded systems, where it powers devices like sensors and microcontrollers, and consumer electronics, including smartphones and smartwatches.
In automotive technology, RISC-V is being integrated into advanced driver-assistance systems (ADAS) and infotainment units. Additionally,
its flexibility makes it a popular choice in academic research for processor design and computer architecture experimentation.
The versatility and open-source model of RISC-V facilitate innovation and customization, driving advancements in numerous technological sectors.

To leverage this flexibility effectively, simulation and validation of custom instructions are essential.
The ETISS (Extendable Translating Instruction Set Simulator) \cite{ETISS} provides a robust platform for instruction set simulation,
offering the capability to model and test custom processor architectures.
However, to fully utilize ETISS's capabilities, it is beneficial to integrate it with a flexible tool for defining custom instructions.

OpenASIP 2.0 \cite{OpenASIP} is a co-design toolset designed to facilitate the customization of RISC-V-based processors,
supporting RTL generation and high-level programming of custom instructions.
By translating OpenASIP's custom instruction sets into CoreDSL \cite{CoreDSL} syntax—a descriptive language for hardware design—we can enhance the integration with ETISS.
CoreDSL allows for a clear and flexible specification of custom instructions,
which can be seamlessly applied to ETISS for comprehensive simulation and evaluation.

Thus, this paper focuses on translating OpenASIP's custom instruction sets into CoreDSL language and applying them within the ETISS framework.
This approach aims to streamline the process of integrating and evaluating custom instructions,
ultimately improving the effectiveness and efficiency of the simulation environment for RISC-V processors.

\appendix
\chapter{Appendix}
\lipsum[4]

% \backmatter
% \bibliography{}

\end{document}
